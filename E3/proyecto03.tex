% --------------------------------------------------------------

\documentclass[spanish, letterpaper,12]{article}

\usepackage[activeacute]{babel}
\usepackage[utf8x]{inputenc}
\usepackage[T1]{fontenc}
\spanishdecimal{.}

\usepackage{amsmath,amsthm,amssymb}
\usepackage[margin=1in]{geometry} 
\usepackage{graphicx}
\usepackage{hyperref}
\usepackage[numbers]{natbib}
\usepackage{enumitem} % para reducir el espacio [noitemsep,nolistsep]
\usepackage{verbatim}

\usepackage{booktabs} % para las tablas \toprule \bo

\DeclareGraphicsExtensions{.pdf,.png,.jpg}
\graphicspath{{./fig/}}


\usepackage{fancyhdr}
\renewcommand{\headrulewidth}{0pt}
\fancyhead[C]{\includegraphics[width=3cm]{upa.jpg}}
\fancyhead[R]{}


\newcommand{\N}{\mathbb{N}}
\newcommand{\Z}{\mathbb{Z}}
 
\newenvironment{theorem}[2][Teorema]{\begin{trivlist}
\item[\hskip \labelsep {\bfseries #1}\hskip \labelsep {\bfseries #2.}]}{\end{trivlist}}
\newenvironment{lemma}[2][Lemma]{\begin{trivlist}
\item[\hskip \labelsep {\bfseries #1}\hskip \labelsep {\bfseries #2.}]}{\end{trivlist}}
\newenvironment{exercise}[2][Ejercicio]{\begin{trivlist}
\item[\hskip \labelsep {\bfseries #1}\hskip \labelsep {\bfseries #2.}]}{\end{trivlist}}
\newenvironment{problem}[2][Problema]{\begin{trivlist}
\item[\hskip \labelsep {\bfseries #1}\hskip \labelsep {\bfseries #2.}]}{\end{trivlist}}
\newenvironment{question}[2][Pregunta]{\begin{trivlist}
\item[\hskip \labelsep {\bfseries #1}\hskip \labelsep {\bfseries #2.}]}{\end{trivlist}}
\newenvironment{corollary}[2][Corollary]{\begin{trivlist}
\item[\hskip \labelsep {\bfseries #1}\hskip \labelsep {\bfseries #2.}]}{\end{trivlist}}
 
\begin{document}
 
% --------------------------------------------------------------
%                         Start here
% --------------------------------------------------------------
 
\title{Proyecto: Construcción de un modelo\\
  de aprovechamiento térmico solar}
\author{Profesor: Dr. Isaías Moreno Cruz\\
  Ingeniería en Energía Fototérmica (2023)\\
Universidad Politécnica de Aguascalientes (UPA)}
\date{01 de diciembre de 2023}

\maketitle

\begin{itemize}[leftmargin=*, noitemsep]
\item \textbf{Fecha de encargo:} viernes 13 de noviembre
\item \textbf{Fecha de entrega:} viernes 01 de diciembre
\end{itemize}
% You can use theorem, exercise, problem, or question here.
% --------------------------------------------------------------
\thispagestyle{fancy}

\section*{Proyecto:}

En el presente proyecto se debe diseñar y construir un modelo de aprovechamiento solar, como lo pueden ser: un reloj solar, cocina solar, un secador solar, etc. 


El reporte debe entregarse impreso y contener:
\begin{itemize}
\item \textbf{Carátula (5\%)}. Información necesaria que identifica al autor y la naturaleza del proyecto, como por ejemplo: Universidad, Materia, Alumno, Grupo, Profesor, Nombre del trabajo, Fecha, etc.
\item \textbf{Introducción (5\%)}. Información general que nos permita introducirnos al tema. La introducción permite ir de los general a lo particular.
\item \textbf{Metodología (10\%)}. Indicar el objetivo del trabajo y documentar, paso a paso, el procedimiento para lograrlo.   
\item \textbf{Resultados (30\%)}. Documentar los resultados obtenidos del objetivo planteado, gráficas o valores numéricos relevantes. Se debe de explicar cada gráfica que se agrega y las gráficas deben de tener las etiquetas que permitan leer correctamente los ejes. Cada Valor documentado debe de tener sus unidades correspondientes.
\item \textbf{Conclusiones (10\%)}. Se debe de redactar el aprendizaje que se tuvo al realizar el trabajo (tomando en cuenta el objetivo), cuales son los resultados más relevantes, las complicaciones se se tuvieron y los trabajos futuros.
\end{itemize}

Adicionalmente al reporte se evaluara la construcción del \textbf{Modelo (40\%)}. El modelo debe de ser funcional y cumplir con el objetivo.

Algunas recomendaciones:
\begin{itemize}
\item Titulo del trabajo: Construcción de un ``nombre del modelo''
\item Objetivo: El presente trabajo pretende demostrar las aplicaciones de la energía solar térmica.
\end{itemize}

El objetivo del modelo será redactado por el alumno y dependerá de lo que pretenda obtener con su modelo, que puede ser: educativo, demostrativo o un prototipo.

Los reportes no deben de exceder las 6 hojas, sin considerar la carátula. Se puede agregar un anexo, con fotografías y/o tablas, estas hojas no estan consideradas dentro de las seis hojas.

\end{document}





